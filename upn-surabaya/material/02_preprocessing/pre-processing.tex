\documentclass[10pt,ignorenonframetext,,aspectratio=149]{beamer}
\usefonttheme{serif} % use mainfont rather than sansfont for slide text
\setbeamertemplate{caption}[numbered]
\setbeamertemplate{caption label separator}{: }
\setbeamercolor{caption name}{fg=normal text.fg}
\usepackage{lmodern}
\usepackage{amssymb,amsmath}
\usepackage{ifxetex,ifluatex}
\usepackage{fixltx2e} % provides \textsubscript
\ifnum 0\ifxetex 1\fi\ifluatex 1\fi=0 % if pdftex
  \usepackage[T1]{fontenc}
  \usepackage[utf8]{inputenc}
\else % if luatex or xelatex
  \ifxetex
    \usepackage{mathspec}
  \else
    \usepackage{fontspec}
  \fi
  \defaultfontfeatures{Ligatures=TeX,Scale=MatchLowercase}
  \newcommand{\euro}{€}
    \setmainfont[]{Open Sans}
\fi
% use upquote if available, for straight quotes in verbatim environments
\IfFileExists{upquote.sty}{\usepackage{upquote}}{}
% use microtype if available
\IfFileExists{microtype.sty}{%
\usepackage{microtype}
\UseMicrotypeSet[protrusion]{basicmath} % disable protrusion for tt fonts
}{}
\usepackage{color}
\usepackage{fancyvrb}
\newcommand{\VerbBar}{|}
\newcommand{\VERB}{\Verb[commandchars=\\\{\}]}
\DefineVerbatimEnvironment{Highlighting}{Verbatim}{commandchars=\\\{\}}
% Add ',fontsize=\small' for more characters per line
\usepackage{framed}
\definecolor{shadecolor}{RGB}{248,248,248}
\newenvironment{Shaded}{\begin{snugshade}}{\end{snugshade}}
\newcommand{\AlertTok}[1]{\textcolor[rgb]{0.94,0.16,0.16}{#1}}
\newcommand{\AnnotationTok}[1]{\textcolor[rgb]{0.56,0.35,0.01}{\textbf{\textit{#1}}}}
\newcommand{\AttributeTok}[1]{\textcolor[rgb]{0.77,0.63,0.00}{#1}}
\newcommand{\BaseNTok}[1]{\textcolor[rgb]{0.00,0.00,0.81}{#1}}
\newcommand{\BuiltInTok}[1]{#1}
\newcommand{\CharTok}[1]{\textcolor[rgb]{0.31,0.60,0.02}{#1}}
\newcommand{\CommentTok}[1]{\textcolor[rgb]{0.56,0.35,0.01}{\textit{#1}}}
\newcommand{\CommentVarTok}[1]{\textcolor[rgb]{0.56,0.35,0.01}{\textbf{\textit{#1}}}}
\newcommand{\ConstantTok}[1]{\textcolor[rgb]{0.00,0.00,0.00}{#1}}
\newcommand{\ControlFlowTok}[1]{\textcolor[rgb]{0.13,0.29,0.53}{\textbf{#1}}}
\newcommand{\DataTypeTok}[1]{\textcolor[rgb]{0.13,0.29,0.53}{#1}}
\newcommand{\DecValTok}[1]{\textcolor[rgb]{0.00,0.00,0.81}{#1}}
\newcommand{\DocumentationTok}[1]{\textcolor[rgb]{0.56,0.35,0.01}{\textbf{\textit{#1}}}}
\newcommand{\ErrorTok}[1]{\textcolor[rgb]{0.64,0.00,0.00}{\textbf{#1}}}
\newcommand{\ExtensionTok}[1]{#1}
\newcommand{\FloatTok}[1]{\textcolor[rgb]{0.00,0.00,0.81}{#1}}
\newcommand{\FunctionTok}[1]{\textcolor[rgb]{0.00,0.00,0.00}{#1}}
\newcommand{\ImportTok}[1]{#1}
\newcommand{\InformationTok}[1]{\textcolor[rgb]{0.56,0.35,0.01}{\textbf{\textit{#1}}}}
\newcommand{\KeywordTok}[1]{\textcolor[rgb]{0.13,0.29,0.53}{\textbf{#1}}}
\newcommand{\NormalTok}[1]{#1}
\newcommand{\OperatorTok}[1]{\textcolor[rgb]{0.81,0.36,0.00}{\textbf{#1}}}
\newcommand{\OtherTok}[1]{\textcolor[rgb]{0.56,0.35,0.01}{#1}}
\newcommand{\PreprocessorTok}[1]{\textcolor[rgb]{0.56,0.35,0.01}{\textit{#1}}}
\newcommand{\RegionMarkerTok}[1]{#1}
\newcommand{\SpecialCharTok}[1]{\textcolor[rgb]{0.00,0.00,0.00}{#1}}
\newcommand{\SpecialStringTok}[1]{\textcolor[rgb]{0.31,0.60,0.02}{#1}}
\newcommand{\StringTok}[1]{\textcolor[rgb]{0.31,0.60,0.02}{#1}}
\newcommand{\VariableTok}[1]{\textcolor[rgb]{0.00,0.00,0.00}{#1}}
\newcommand{\VerbatimStringTok}[1]{\textcolor[rgb]{0.31,0.60,0.02}{#1}}
\newcommand{\WarningTok}[1]{\textcolor[rgb]{0.56,0.35,0.01}{\textbf{\textit{#1}}}}

% Comment these out if you don't want a slide with just the
% part/section/subsection/subsubsection title:
\AtBeginPart{
  \let\insertpartnumber\relax
  \let\partname\relax
  \frame{\partpage}
}
\AtBeginSection{
  \let\insertsectionnumber\relax
  \let\sectionname\relax
  \frame{\sectionpage}
}
\AtBeginSubsection{
  \let\insertsubsectionnumber\relax
  \let\subsectionname\relax
  \frame{\subsectionpage}
}

\setlength{\emergencystretch}{3em}  % prevent overfull lines
\providecommand{\tightlist}{%
  \setlength{\itemsep}{0pt}\setlength{\parskip}{0pt}}
\setcounter{secnumdepth}{0}

\title{Pre-processing data di R}
\subtitle{(Pelatihan data sains menggunakan R dan Gephi)}
\author{Ujang Fahmi}
\date{}

%% Here's everything I added.
%%--------------------------

\usepackage{graphicx}
\usepackage{rotating}
%\setbeamertemplate{caption}[numbered]
\usepackage{hyperref}
\usepackage{caption}
\usepackage[normalem]{ulem}
%\mode<presentation>
\usepackage{wasysym}
%\usepackage{amsmath}


% Get rid of navigation symbols.
%-------------------------------
\setbeamertemplate{navigation symbols}{}

% Optional institute tags and titlegraphic.
% Do feel free to change the titlegraphic if you don't want it as a Markdown field.
%----------------------------------------------------------------------------------
\institute{Pelajaran ke-2}

% \titlegraphic{\includegraphics[width=0.3\paperwidth]{\string~/Dropbox/teaching/clemson-academic.png}} % <-- if you want to know what this looks like without it as a Markdown field. 
% -----------------------------------------------------------------------------------------------------
\titlegraphic{\includegraphics[width=0.3\paperwidth]{styles/sadasa.png}}

% Some additional title page adjustments.
%----------------------------------------
\setbeamertemplate{title page}[empty]
%\date{}
\setbeamerfont{subtitle}{size=\small}

\setbeamercovered{transparent}

% Some optional colors. Change or add as you see fit.
%---------------------------------------------------
\definecolor{clemsonpurple}{HTML}{522D80}
 \definecolor{clemsonorange}{HTML}{F66733}
\definecolor{uiucblue}{HTML}{003C7D}
\definecolor{uiucorange}{HTML}{F47F24}


% Some optional color adjustments to Beamer. Change as you see fit.
%------------------------------------------------------------------
\setbeamercolor{frametitle}{fg=clemsonpurple,bg=white}
\setbeamercolor{title}{fg=clemsonpurple,bg=white}
\setbeamercolor{local structure}{fg=clemsonpurple}
\setbeamercolor{section in toc}{fg=clemsonpurple,bg=white}
% \setbeamercolor{subsection in toc}{fg=clemsonorange,bg=white}
\setbeamercolor{footline}{fg=clemsonpurple!50, bg=white}
\setbeamercolor{block title}{fg=clemsonorange,bg=white}


\let\Tiny=\tiny


% Sections and subsections should not get their own damn slide.
%--------------------------------------------------------------
\AtBeginPart{}
\AtBeginSection{}
\AtBeginSubsection{}
\AtBeginSubsubsection{}

% Suppress some of Markdown's weird default vertical spacing.
%------------------------------------------------------------
\setlength{\emergencystretch}{0em}  % prevent overfull lines
\setlength{\parskip}{0pt}


% Allow for those simple two-tone footlines I like. 
% Edit the colors as you see fit.
%--------------------------------------------------
\defbeamertemplate*{footline}{my footline}{%
    \ifnum\insertpagenumber=1
    \hbox{%
        \begin{beamercolorbox}[wd=\paperwidth,ht=.8ex,dp=1ex,center]{}%
      % empty environment to raise height
        \end{beamercolorbox}%
    }%
    \vskip0pt%
    \else%
        \Tiny{%
            \hfill%
		\vspace*{1pt}%
            \insertframenumber/\inserttotalframenumber \hspace*{0.1cm}%
            \newline%
            \color{clemsonpurple}{\rule{\paperwidth}{0.4mm}}\newline%
            \color{clemsonorange}{\rule{\paperwidth}{.4mm}}%
        }%
    \fi%
}

% Various cosmetic things, though I must confess I forget what exactly these do and why I included them.
%-------------------------------------------------------------------------------------------------------
\setbeamercolor{structure}{fg=blue}
\setbeamercolor{local structure}{parent=structure}
\setbeamercolor{item projected}{parent=item,use=item,fg=clemsonpurple,bg=white}
\setbeamercolor{enumerate item}{parent=item}

% Adjust some item elements. More cosmetic things.
%-------------------------------------------------
\setbeamertemplate{itemize item}{\color{clemsonpurple}$\bullet$}
\setbeamertemplate{itemize subitem}{\color{clemsonpurple}\scriptsize{$\bullet$}}
\setbeamertemplate{itemize/enumerate body end}{\vspace{.6\baselineskip}} % So I'm less inclined to use \medskip and \bigskip in Markdown.

% Automatically center images
% ---------------------------
% Note: this is for ![](image.png) images
% Use "fig.align = "center" for R chunks

\usepackage{etoolbox}

\AtBeginDocument{%
  \letcs\oig{@orig\string\includegraphics}%
  \renewcommand<>\includegraphics[2][]{%
    \only#3{%
      {\centering\oig[{#1}]{#2}\par}%
    }%
  }%
}

% I think I've moved to xelatex now. Here's some stuff for that.
% --------------------------------------------------------------
% I could customize/generalize this more but the truth is it works for my circumstances.

\ifxetex
\setbeamerfont{title}{family=\fontspec{Titillium Web}}
\setbeamerfont{frametitle}{family=\fontspec{Titillium Web}}
\usepackage[font=small,skip=0pt]{caption}
 \else
 \fi

% Okay, and begin the actual document...

\begin{document}
\frame{\titlepage}

\begin{frame}
Salam kenal dan selamat datang.

Semoga kita semua bisa saling berbagi pengalaman dan pengetahuan. Saya
adalah Ujang Fahmi, Co-founder dan mentor Sadasa Academy.

\vspace{0.1in}

Jika anda berada dan sedang membaca tutorial ini, maka kemungkinan anda
adalah orang yang sedang ingin belajar data sains, atau mungkin
ditugaskan untuk mempelajari R oleh institusi atau organisasi anda. Sama
seperti saya dulu, dimana tanpa latar belakang enginering saya
didiharuskan untuk belajar R, demi menyelesaikan tugas akhir dan
akhirnya jadilah seperti saya sekarang ini.

\vspace{0.1in}

Satu hal yang pasti, ini adalah langkah pertama dari banyak langkah yang
harus dilalui, entah melalui lembaga resmi atau belajar secara mandiri.
Jadi selamat belajar!!!

\vspace{0.1in}

Ujang Fahmi,

Yogyakarta, 2021-09-17
\end{frame}

\hypertarget{mendapatkan-bantuan-dan-mengatasi-error}{%
\section{Mendapatkan Bantuan dan Mengatasi
Error}\label{mendapatkan-bantuan-dan-mengatasi-error}}

\hypertarget{error-yang-umum-1}{%
\subsection{Error yang umum 1}\label{error-yang-umum-1}}

\begin{frame}[fragile]{Error yang umum 1}
Di worksapce kita menulis

\begin{Shaded}
\begin{Highlighting}[]
\NormalTok{a }\SpecialCharTok{+} \DecValTok{198}
\end{Highlighting}
\end{Shaded}

Di console menampilkan:

\texttt{Error:\ object\ \textquotesingle{}a\textquotesingle{}\ not\ found}

Artinya objek \texttt{a} tidak bisa ditemukan sehingga R tidak bisa
menyelesaikan perintah yang diberikan
\end{frame}

\hypertarget{error-yang-umum-2}{%
\subsection{Error yang umum 2}\label{error-yang-umum-2}}

\begin{frame}[fragile]{Error yang umum 2}
Di workspace kita mencoba untuk memberikan perintah impor data dengan
menggunakan fungsi \texttt{read\_csv}, kita juga udah yakin argumen yang
dibutuhkan sudah sesuai.

\begin{Shaded}
\begin{Highlighting}[]
\NormalTok{df1 }\OtherTok{=} \FunctionTok{read\_csv}\NormalTok{(}\StringTok{"folder1/file.csv"}\NormalTok{)}
\end{Highlighting}
\end{Shaded}

Tapi, di \texttt{console} kita masih tetap menerima error berikut:

\texttt{Error\ in\ read\_csv("folder1/file.csv")\ :}

\texttt{could\ not\ find\ function\ "read\_csv"}

Error menunjukkan bahwa fungsi \texttt{read\_csv} tidak ditemukan.
Artinya kita belum memanggil library yang dibutuhkan.
\end{frame}

\hypertarget{impor-dan-ekspor-data}{%
\section{Impor dan Ekspor Data}\label{impor-dan-ekspor-data}}

\begin{frame}[fragile]{Impor dan Ekspor Data}
\begin{quote}
Di R kita hanya bisa mengolah data yang yang sudah kita impor atau
berada di \emph{environment} R. Maka ketika ada
\texttt{error:\ object\ not\ found} di console, sebagian besar
disebabkan oleh data/objek yang belum ada.
\end{quote}
\end{frame}

\hypertarget{impor-data}{%
\subsection{Impor Data}\label{impor-data}}

\begin{frame}[fragile]{Impor Data}
\begin{itemize}
\tightlist
\item
  Impor data merupakan salah satu langkah pertama dan utama dalam
  pengolahan data
\item
  impor umumnya menggunakan sintak dengan awalan \texttt{read}. Misal
  \texttt{read.csv()} atau \texttt{read\_csv}, sama-sama digunakan untuk
  impor data \texttt{.csv}
\end{itemize}
\end{frame}

\begin{frame}[fragile]{Impor data frame}
\protect\hypertarget{impor-data-frame}{}
Data frame atau data flate di R bisa diimpor dengan beberapa sintaks
dari beberapa package.

Argumen yang dibutuhkan:

\begin{enumerate}
\tightlist
\item
  namfolder
\item
  namafile atau link data
\end{enumerate}

\begin{Shaded}
\begin{Highlighting}[]
\CommentTok{\# dasar}
\NormalTok{df1 }\OtherTok{=} \FunctionTok{read.csv}\NormalTok{(}\AttributeTok{file =} \StringTok{"namafolder/namafile.csv"}\NormalTok{, }\AttributeTok{header =} \ConstantTok{TRUE}\NormalTok{, }
               \AttributeTok{sep =} \StringTok{","}\NormalTok{)}
\CommentTok{\# readr}
\FunctionTok{library}\NormalTok{(readr)}
\NormalTok{df2 }\OtherTok{=} \FunctionTok{read\_csv}\NormalTok{(}\AttributeTok{file =} \StringTok{"namafolder/namafile.csv"}\NormalTok{, }\AttributeTok{trim\_ws =} \ConstantTok{TRUE}\NormalTok{)}
\CommentTok{\# xlsx}
\FunctionTok{library}\NormalTok{(readxl)}
\NormalTok{df3 }\OtherTok{=} \FunctionTok{read\_excel}\NormalTok{(}\AttributeTok{path =} \StringTok{"namafolder/namafile.xlsx"}\NormalTok{, }\AttributeTok{sheet =} \DecValTok{1}\NormalTok{)}
\end{Highlighting}
\end{Shaded}
\end{frame}

\begin{frame}[fragile]{Impor data json}
\protect\hypertarget{impor-data-json}{}
\texttt{Json} atau \texttt{java\ script\ object\ notation} adalah format
data yang umum digunakan untuk ditampilkan di sebuah laman/website.

Contoh:

\begin{quote}
\url{https://raw.githubusercontent.com/eppofahmi/belajaR/master/upn-surabaya/data/sample.json}
\end{quote}

\begin{Shaded}
\begin{Highlighting}[]
\FunctionTok{library}\NormalTok{(jsonlite)}

\NormalTok{df4 }\OtherTok{=} \FunctionTok{fromJSON}\NormalTok{(}\StringTok{"namafolder/namafile.json"}\NormalTok{,}
               \AttributeTok{simplifyDataFrame =} \ConstantTok{TRUE}\NormalTok{)}
\end{Highlighting}
\end{Shaded}

Sama seperi data lainnya, untuk mengimpor data \texttt{json} kita perlu
menulis nama folder dan nama filenya. Selain data yang ada di folder
dalam komputer, kita juga bisa membaca data yang ada di cloud atau
website dengan ciri terdapta ekstensi data di url-nya seperti contoh di
atas.
\end{frame}

\begin{frame}[fragile]{Impor data lain}
\protect\hypertarget{impor-data-lain}{}
Selain jenis-jenis format data seperti yang sudah dibahas. Di R kita
juga bisa membaca jenis lain, seperti:

\begin{enumerate}
\tightlist
\item
  Data output dari SPSS dan Stata
\item
  Data \texttt{.nc}
\item
  Data list
\item
  etc.
\end{enumerate}

TIPS:

Jika ingin mengimpor data dengan ekstensi tertentu, kita bisa mencari
tutorialnya dengan mengetikan di google search. Misalnya untuk membaca
file stata, kita bisa menggunakan kata kunci berikut.

\begin{quote}
read stata data in r
\end{quote}
\end{frame}

\hypertarget{your-turn-1}{%
\subsection{Your Turn 1}\label{your-turn-1}}

\begin{frame}{Your Turn 1}
\begin{enumerate}
\tightlist
\item
  Buatlah data xlsx dan csv dengan menggunakan excel
\item
  Letakan di folder data
\item
  Impor data tersebut menggunakan sintak-sintak yang sudah dipelajari
\item
  Impor data json dari url yang ada dicontoh
\end{enumerate}
\end{frame}

\hypertarget{ekspor-data}{%
\subsection{Ekspor Data}\label{ekspor-data}}

\begin{frame}{Ekspor Data}
Setelah bisa mengimpor, dan setelah melakukan pemrosesan di R,
selanjutnya kita mungkin perlu menyimpan data dihardisk.
\end{frame}

\begin{frame}[fragile]{Ekspor data frame}
\protect\hypertarget{ekspor-data-frame}{}
Jika untuk mengimpor sebagian besar fungsinya diawali dengan
\texttt{read*} maka untuk mengekspor data sebagian besar fungsinya
diawali dengan \texttt{write}.

Contoh:

\begin{Shaded}
\begin{Highlighting}[]
\FunctionTok{write\_csv}\NormalTok{(namaobjek, }\StringTok{"namafolder/namafile.csv"}\NormalTok{)}
\end{Highlighting}
\end{Shaded}
\end{frame}

\begin{frame}[fragile]{Ekspor list}
\protect\hypertarget{ekspor-list}{}
List atau bisa juga disebut sebagai data bertingkat juga dapat disimpan
atau diekspor dari R. Tipe data ini biasanya digunakan untuk menyimpan
data hasil pengolahan dalam satu file. Untuk mengekspor list di R
umumnya dilakukan menggunakan tipe data \texttt{.rds}. Tipe data ini
hanya spesifik untuk R saja.

\vspace{0.1in}

Jika Menghendaki format yang lebih umum, maka sebaiknya list disimpan
dalam format data yang juga umum. Misalnya json.

\begin{Shaded}
\begin{Highlighting}[]
\FunctionTok{library}\NormalTok{(readr)}

\FunctionTok{write\_rds}\NormalTok{(namaobjek, }\StringTok{"namafolder/namafile.rds"}\NormalTok{)}
\end{Highlighting}
\end{Shaded}
\end{frame}

\begin{frame}[fragile]{Ekspor json}
\protect\hypertarget{ekspor-json}{}
Untuk menyimpan json kita juga bisa menggunakan pola sintak yang sama
seperti yang sudah dipelajari sebelumnya, yaitu:

\begin{Shaded}
\begin{Highlighting}[]
\NormalTok{jsonlite}\SpecialCharTok{::}\FunctionTok{write\_json}\NormalTok{(}\AttributeTok{x =}\NormalTok{ namaobjek, }\AttributeTok{path =} \StringTok{"namafolder/namafile.json"}\NormalTok{)}
\end{Highlighting}
\end{Shaded}
\end{frame}

\hypertarget{pengecekan-data}{%
\section{Pengecekan Data}\label{pengecekan-data}}

\begin{frame}[fragile]{Pengecekan Data}
\begin{quote}
Cobalah untuk mengimpor data ke R, dan perhatikan apa yang ada dibagian
environment.
\end{quote}

\begin{Shaded}
\begin{Highlighting}[]
\NormalTok{nama\_objek1 }\OtherTok{=} \FunctionTok{read\_csv}\NormalTok{(}\StringTok{"folder/file.csv"}\NormalTok{)}
\end{Highlighting}
\end{Shaded}

Gantilah nama folder dan file yang sesuai dengan yang dimiliki di laptop
masing-masing
\end{frame}

\hypertarget{jumlah-observasi-dan-variabel}{%
\subsection{Jumlah observasi dan
variabel}\label{jumlah-observasi-dan-variabel}}

\begin{frame}[fragile]{Jumlah observasi dan variabel}
\begin{columns}[T]
\begin{column}{0.5\textwidth}
\begin{itemize}
\tightlist
\item
  Jumlah observasi menunjukkan jumlah baris, sementara jumlah variabel
  menunjukkan jumlah kolom
\item
  Selain itu, sebagai analis kita juga harus tahu masing-masing tipe
  variabel yang dimiliki
\item
  Untuk mengetahuinya kita bisa menggunakan fungsi glimpse()
\end{itemize}
\end{column}

\begin{column}{0.5\textwidth}
Contoh:

\begin{Shaded}
\begin{Highlighting}[]
\NormalTok{nama\_objek2 }\OtherTok{=}\NormalTok{ mtcars}
\FunctionTok{glimpse}\NormalTok{(nama\_objek2)}
\end{Highlighting}
\end{Shaded}

\begin{verbatim}
## Rows: 32
## Columns: 11
## $ mpg  <dbl> 21.0, 21.0, 22.8, 21.4, 18.7, 18.1, 14.3, 24.4, 22.8, 19.2, 17.8,~
## $ cyl  <dbl> 6, 6, 4, 6, 8, 6, 8, 4, 4, 6, 6, 8, 8, 8, 8, 8, 8, 4, 4, 4, 4, 8,~
## $ disp <dbl> 160.0, 160.0, 108.0, 258.0, 360.0, 225.0, 360.0, 146.7, 140.8, 16~
## $ hp   <dbl> 110, 110, 93, 110, 175, 105, 245, 62, 95, 123, 123, 180, 180, 180~
## $ drat <dbl> 3.90, 3.90, 3.85, 3.08, 3.15, 2.76, 3.21, 3.69, 3.92, 3.92, 3.92,~
## $ wt   <dbl> 2.620, 2.875, 2.320, 3.215, 3.440, 3.460, 3.570, 3.190, 3.150, 3.~
## $ qsec <dbl> 16.46, 17.02, 18.61, 19.44, 17.02, 20.22, 15.84, 20.00, 22.90, 18~
## $ vs   <dbl> 0, 0, 1, 1, 0, 1, 0, 1, 1, 1, 1, 0, 0, 0, 0, 0, 0, 1, 1, 1, 1, 0,~
## $ am   <dbl> 1, 1, 1, 0, 0, 0, 0, 0, 0, 0, 0, 0, 0, 0, 0, 0, 0, 1, 1, 1, 0, 0,~
## $ gear <dbl> 4, 4, 4, 3, 3, 3, 3, 4, 4, 4, 4, 3, 3, 3, 3, 3, 3, 4, 4, 4, 3, 3,~
## $ carb <dbl> 4, 4, 1, 1, 2, 1, 4, 2, 2, 4, 4, 3, 3, 3, 4, 4, 4, 1, 2, 1, 1, 2,~
\end{verbatim}
\end{column}
\end{columns}
\end{frame}

\hypertarget{rangkuman-data}{%
\subsection{Rangkuman data}\label{rangkuman-data}}

\begin{frame}[fragile]{Rangkuman data}
\begin{columns}[T]
\begin{column}{0.5\textwidth}
\begin{itemize}
\tightlist
\item
  Untuk mendapatkan rangkuman dari data kita bisa menggunakan fungsi
  \texttt{summary()} atau \texttt{skim()} dari package \texttt{skimr}
\item
  Rangkuman biasanya digunakan untuk mendapatkan informasi umum terkait
  dengan data yang akan diolah
\item
  Rangkuman secara umum jika angka akan berupa \emph{central tendency}
  statistik
\item
  Rangkuman data teks biasanya berupa jenis datanya saja
\end{itemize}
\end{column}

\begin{column}{0.5\textwidth}
Contoh:

\begin{Shaded}
\begin{Highlighting}[]
\FunctionTok{summary}\NormalTok{(nama\_objek2)}
\end{Highlighting}
\end{Shaded}

\begin{verbatim}
##       mpg             cyl             disp             hp       
##  Min.   :10.40   Min.   :4.000   Min.   : 71.1   Min.   : 52.0  
##  1st Qu.:15.43   1st Qu.:4.000   1st Qu.:120.8   1st Qu.: 96.5  
##  Median :19.20   Median :6.000   Median :196.3   Median :123.0  
##  Mean   :20.09   Mean   :6.188   Mean   :230.7   Mean   :146.7  
##  3rd Qu.:22.80   3rd Qu.:8.000   3rd Qu.:326.0   3rd Qu.:180.0  
##  Max.   :33.90   Max.   :8.000   Max.   :472.0   Max.   :335.0  
##       drat             wt             qsec             vs        
##  Min.   :2.760   Min.   :1.513   Min.   :14.50   Min.   :0.0000  
##  1st Qu.:3.080   1st Qu.:2.581   1st Qu.:16.89   1st Qu.:0.0000  
##  Median :3.695   Median :3.325   Median :17.71   Median :0.0000  
##  Mean   :3.597   Mean   :3.217   Mean   :17.85   Mean   :0.4375  
##  3rd Qu.:3.920   3rd Qu.:3.610   3rd Qu.:18.90   3rd Qu.:1.0000  
##  Max.   :4.930   Max.   :5.424   Max.   :22.90   Max.   :1.0000  
##        am              gear            carb      
##  Min.   :0.0000   Min.   :3.000   Min.   :1.000  
##  1st Qu.:0.0000   1st Qu.:3.000   1st Qu.:2.000  
##  Median :0.0000   Median :4.000   Median :2.000  
##  Mean   :0.4062   Mean   :3.688   Mean   :2.812  
##  3rd Qu.:1.0000   3rd Qu.:4.000   3rd Qu.:4.000  
##  Max.   :1.0000   Max.   :5.000   Max.   :8.000
\end{verbatim}
\end{column}
\end{columns}
\end{frame}

\hypertarget{penataan-data-data-wrangling}{%
\section{\texorpdfstring{Penataan Data \emph{(Data
Wrangling)}}{Penataan Data (Data Wrangling)}}\label{penataan-data-data-wrangling}}

\begin{frame}{Penataan Data \emph{(Data Wrangling)}}
Penataan atau juga bisa disebut wrangling data juga merupakan salah satu
bagian penting dalam tahap pra-pemrosesan data. Hal ini bisanya
dilakukan mulai dari menstandarisasi nama kolom hingga memutuskan metode
pengisian data yang kosong atau imputasi.

\vspace{0.1in}

Data wrangling berfungsi agar analis bisa fokus pada metode dan hasil
pada saat melakukan analisis. Tidak lagi bingung atau harus menata data
yang tidak terstruktur yang tidak mudah dipahami.
\end{frame}

\hypertarget{standar-nama-kolom}{%
\subsection{Standar nama kolom}\label{standar-nama-kolom}}

\begin{frame}[fragile]{Standar nama kolom}
\begin{columns}[T]
\begin{column}{0.5\textwidth}
Nama kolom/variabel di R sebaiknya dibuat berdasarkan aturan yang sama
untuk membuat objek R, yaitu:

\begin{itemize}
\tightlist
\item
  Tidak diawali dengan angka (\texttt{kolom1}, bukan \texttt{1kolom})
\item
  Tidak menggunakan spasi (\texttt{kolom\_ke1} bukan
  \texttt{kolom\ ke\ 1})
\item
  Sebaiknya menggunakan huruf yang seragam (huruf kecil atau besar
  semua)
\end{itemize}
\end{column}

\begin{column}{0.5\textwidth}
Contoh:

\begin{Shaded}
\begin{Highlighting}[]
\FunctionTok{library}\NormalTok{(janitor)}
\CommentTok{\# impor data}
\NormalTok{df1 }\OtherTok{\textless{}{-}} \FunctionTok{read\_delim}\NormalTok{(}\StringTok{"data/sample.csv"}\NormalTok{, }\StringTok{";"}\NormalTok{, }
    \AttributeTok{escape\_double =} \ConstantTok{FALSE}\NormalTok{)}
\FunctionTok{glimpse}\NormalTok{(df1)}

\CommentTok{\# nama kolom standar}
\NormalTok{df1 }\OtherTok{=}\NormalTok{ janitor}\SpecialCharTok{::}\FunctionTok{clean\_names}\NormalTok{(df1)}
\FunctionTok{glimpse}\NormalTok{(df1)}
\end{Highlighting}
\end{Shaded}
\end{column}
\end{columns}
\end{frame}

\hypertarget{memilih-kolom-dan-menyusun-nama-kolom}{%
\subsection{Memilih Kolom dan Menyusun nama
kolom}\label{memilih-kolom-dan-menyusun-nama-kolom}}

\begin{frame}[fragile]{Memilih Kolom dan Menyusun nama kolom}
\begin{columns}[T]
\begin{column}{0.5\textwidth}
Untuk memilih nama kolom kita bisa menggunakan fungsi \texttt{select}
dari \texttt{dplyr}. Namun sebelum menggunakannya, cobalah untuk
menjalankan dan membaca dokumentasinya dengan menjalankan fungsi
berikut:

\begin{Shaded}
\begin{Highlighting}[]
\FunctionTok{library}\NormalTok{(dplyr)}

\NormalTok{?dplyr}\SpecialCharTok{::}\NormalTok{select}
\end{Highlighting}
\end{Shaded}
\end{column}

\begin{column}{0.5\textwidth}
Untuk menata atau mengurutkan kolom/variabel sesuai kebutuhan kita juga
bisa mengkombinasikan fungsi \texttt{select()} dengan nomor indeks atau
anam kolomnya.

Misalnya:

\begin{Shaded}
\begin{Highlighting}[]
\NormalTok{nama\_objek2 }\SpecialCharTok{\%\textgreater{}\%} 
   \FunctionTok{select}\NormalTok{(}\FunctionTok{c}\NormalTok{(}\DecValTok{1}\SpecialCharTok{:}\DecValTok{4}\NormalTok{, }\DecValTok{10}\NormalTok{))}
\end{Highlighting}
\end{Shaded}

Kita memilih kolom 1 sampai 4 dan kolom ke 10 dari data
\texttt{nama\_objek2}
\end{column}
\end{columns}
\end{frame}

\hypertarget{memilih-baris}{%
\subsection{Memilih Baris}\label{memilih-baris}}

\begin{frame}[fragile]{Memilih Baris}
\begin{itemize}
\tightlist
\item
  Untuk memilih baris kita bisa menggunakan \texttt{filter()} dari dplyr
\item
  \texttt{filter()} digunakan untuk memilih baris dari data berdasarkan
  kondisi yang ditetapkan pada satu atau lebih kolom
\end{itemize}

\begin{Shaded}
\begin{Highlighting}[]
\NormalTok{nama\_objek2 }\SpecialCharTok{\%\textgreater{}\%} 
   \FunctionTok{filter}\NormalTok{(disp }\SpecialCharTok{\textgreater{}} \DecValTok{250}\NormalTok{)}
\end{Highlighting}
\end{Shaded}

Skrip diatas digunakan untuk memfilter (memilih baris) yang nilai
dikolom \texttt{disp} nya lebih dari 250.
\end{frame}

\hypertarget{transformasi-kolom}{%
\subsection{Transformasi kolom}\label{transformasi-kolom}}

\begin{frame}[fragile]{Nama baris menjadi kolom}
\protect\hypertarget{nama-baris-menjadi-kolom}{}
\begin{columns}[T]
\begin{column}{0.5\textwidth}
\begin{Shaded}
\begin{Highlighting}[]
\NormalTok{df3 }\OtherTok{=}\NormalTok{ mtcars}
\NormalTok{df3 }\SpecialCharTok{\%\textgreater{}\%} 
   \FunctionTok{select}\NormalTok{(}\DecValTok{1}\SpecialCharTok{:}\DecValTok{2}\NormalTok{) }\SpecialCharTok{\%\textgreater{}\%} 
   \FunctionTok{head}\NormalTok{(}\DecValTok{5}\NormalTok{)}
\end{Highlighting}
\end{Shaded}

\begin{verbatim}
##                    mpg cyl
## Mazda RX4         21.0   6
## Mazda RX4 Wag     21.0   6
## Datsun 710        22.8   4
## Hornet 4 Drive    21.4   6
## Hornet Sportabout 18.7   8
\end{verbatim}
\end{column}

\begin{column}{0.5\textwidth}
\begin{Shaded}
\begin{Highlighting}[]
\NormalTok{df3 }\OtherTok{=} \FunctionTok{rownames\_to\_column}\NormalTok{(df3)}
\NormalTok{df3 }\SpecialCharTok{\%\textgreater{}\%} 
   \FunctionTok{select}\NormalTok{(}\DecValTok{1}\SpecialCharTok{:}\DecValTok{2}\NormalTok{) }\SpecialCharTok{\%\textgreater{}\%} 
   \FunctionTok{head}\NormalTok{(}\DecValTok{5}\NormalTok{)}
\end{Highlighting}
\end{Shaded}

\begin{verbatim}
##             rowname  mpg
## 1         Mazda RX4 21.0
## 2     Mazda RX4 Wag 21.0
## 3        Datsun 710 22.8
## 4    Hornet 4 Drive 21.4
## 5 Hornet Sportabout 18.7
\end{verbatim}
\end{column}
\end{columns}
\end{frame}

\begin{frame}[fragile]{Memisahkan nilai kolom}
\protect\hypertarget{memisahkan-nilai-kolom}{}
Untuk memisahkan nilai kolom kita bisa menggunakan fungsi
\texttt{separate()}

\begin{Shaded}
\begin{Highlighting}[]
\NormalTok{df4 }\OtherTok{=}\NormalTok{ msleep}
\NormalTok{df4 }\SpecialCharTok{\%\textgreater{}\%} 
   \FunctionTok{select}\NormalTok{(}\DecValTok{1}\SpecialCharTok{:}\DecValTok{3}\NormalTok{) }\SpecialCharTok{\%\textgreater{}\%} 
   \FunctionTok{head}\NormalTok{(}\DecValTok{5}\NormalTok{)}
\end{Highlighting}
\end{Shaded}

Menjadi\ldots{}

\begin{Shaded}
\begin{Highlighting}[]
\NormalTok{df4 }\OtherTok{=}\NormalTok{ msleep}
\NormalTok{df4 }\OtherTok{=}\NormalTok{ df4 }\SpecialCharTok{\%\textgreater{}\%} 
   \FunctionTok{separate}\NormalTok{(name, }\AttributeTok{into =} \FunctionTok{c}\NormalTok{(}\StringTok{"nama\_depan"}\NormalTok{, }
                           \StringTok{"nama\_tengah"}\NormalTok{, }
                           \StringTok{"nama\_belakang"}\NormalTok{), }
            \AttributeTok{sep =} \StringTok{" "}\NormalTok{) }\SpecialCharTok{\%\textgreater{}\%} 
   \FunctionTok{select}\NormalTok{(}\DecValTok{1}\SpecialCharTok{:}\DecValTok{3}\NormalTok{) }\SpecialCharTok{\%\textgreater{}\%} 
   \FunctionTok{head}\NormalTok{(}\DecValTok{5}\NormalTok{)}
\end{Highlighting}
\end{Shaded}
\end{frame}

\begin{frame}[fragile]{Menggabungkan nilai kolom}
\protect\hypertarget{menggabungkan-nilai-kolom}{}
Untuk mengabungkan nilai beberapa kolom menjadi berada dalam satu kolom
kita bisa menggunakan fungsi \texttt{unite()}

\begin{Shaded}
\begin{Highlighting}[]
\NormalTok{df4 }\SpecialCharTok{\%\textgreater{}\%} 
   \FunctionTok{unite}\NormalTok{(}\AttributeTok{col =} \StringTok{"nama\_lengkap"}\NormalTok{, }\AttributeTok{sep =} \StringTok{" "}\NormalTok{, }\AttributeTok{remove =} \ConstantTok{FALSE}\NormalTok{)}
\end{Highlighting}
\end{Shaded}
\end{frame}

\begin{frame}[fragile]{Membuat kolom baru dan Merangkum data}
\protect\hypertarget{membuat-kolom-baru-dan-merangkum-data}{}
Untuk membuat kolom baru kita bisa menggunakan fungsi \texttt{mutate()},
di sini kita akan menggunakan dengan mengisi jumlah observasi dari kolom
lain.

\begin{Shaded}
\begin{Highlighting}[]
\NormalTok{fauna }\OtherTok{=}\NormalTok{ msleep}
\NormalTok{rangkuman }\OtherTok{=}\NormalTok{ fauna }\SpecialCharTok{\%\textgreater{}\%} 
   \FunctionTok{group\_by}\NormalTok{(vore) }\SpecialCharTok{\%\textgreater{}\%} 
   \FunctionTok{count}\NormalTok{(order)}
\NormalTok{rangkuman}

\NormalTok{mobil }\OtherTok{=}\NormalTok{ mtcars}
\NormalTok{mobil }\SpecialCharTok{\%\textgreater{}\%} 
   \FunctionTok{group\_by}\NormalTok{(cyl) }\SpecialCharTok{\%\textgreater{}\%} 
   \FunctionTok{mutate}\NormalTok{(}\AttributeTok{mean =} \FunctionTok{mean}\NormalTok{(disp)) }\SpecialCharTok{\%\textgreater{}\%} 
   \FunctionTok{select}\NormalTok{(cyl, disp, mean)}
\end{Highlighting}
\end{Shaded}
\end{frame}

\hypertarget{mengatasasi-nilai-hilang}{%
\section{Mengatasasi Nilai Hilang}\label{mengatasasi-nilai-hilang}}

\begin{frame}[fragile]{Mengatasasi Nilai Hilang}
\begin{itemize}
\item
  Terkadang kita memiliki data yang tidak semua barisnya terisi. Kondisi
  tersebut disebut value not available, missinf value atau \texttt{NA}.
\item
  Keputusan untuk tetap menggunakan data yang terdapat nilai kosong
  dalam salah satu variabelnya merupakan keputusan teoretis
\item
  Secara teknis, ada beberapa upaya yang bisa dilakukan. Upaya ini
  disebut imputasi.
\end{itemize}

Anggap saja kita memiliki data berikut, dimana tidak ada nilai yang
kosong di dalamnya.

\begin{Shaded}
\begin{Highlighting}[]
\FunctionTok{library}\NormalTok{(missForest)}
\FunctionTok{data}\NormalTok{(iris)}
\FunctionTok{summary}\NormalTok{(iris)}
\end{Highlighting}
\end{Shaded}

\begin{verbatim}
##   Sepal.Length    Sepal.Width     Petal.Length    Petal.Width   
##  Min.   :4.300   Min.   :2.000   Min.   :1.000   Min.   :0.100  
##  1st Qu.:5.100   1st Qu.:2.800   1st Qu.:1.600   1st Qu.:0.300  
##  Median :5.800   Median :3.000   Median :4.350   Median :1.300  
##  Mean   :5.843   Mean   :3.057   Mean   :3.758   Mean   :1.199  
##  3rd Qu.:6.400   3rd Qu.:3.300   3rd Qu.:5.100   3rd Qu.:1.800  
##  Max.   :7.900   Max.   :4.400   Max.   :6.900   Max.   :2.500  
##        Species  
##  setosa    :50  
##  versicolor:50  
##  virginica :50  
##                 
##                 
## 
\end{verbatim}
\end{frame}

\hypertarget{menggunakan-teknik-manual}{%
\subsection{Menggunakan teknik manual}\label{menggunakan-teknik-manual}}

\begin{frame}[fragile]{Membuat nilai kosong untuk simulasi}
\protect\hypertarget{membuat-nilai-kosong-untuk-simulasi}{}
\begin{Shaded}
\begin{Highlighting}[]
\FunctionTok{set.seed}\NormalTok{(}\DecValTok{1234}\NormalTok{)}
\NormalTok{iris.mis }\OtherTok{\textless{}{-}} \FunctionTok{prodNA}\NormalTok{(iris, }\AttributeTok{noNA =} \FloatTok{0.2}\NormalTok{)}
\FunctionTok{summary}\NormalTok{(iris.mis)}
\end{Highlighting}
\end{Shaded}

\begin{verbatim}
##   Sepal.Length    Sepal.Width     Petal.Length    Petal.Width   
##  Min.   :4.300   Min.   :2.000   Min.   :1.000   Min.   :0.100  
##  1st Qu.:5.100   1st Qu.:2.800   1st Qu.:1.525   1st Qu.:0.300  
##  Median :5.700   Median :3.000   Median :4.400   Median :1.300  
##  Mean   :5.774   Mean   :3.072   Mean   :3.756   Mean   :1.186  
##  3rd Qu.:6.400   3rd Qu.:3.400   3rd Qu.:5.100   3rd Qu.:1.800  
##  Max.   :7.900   Max.   :4.400   Max.   :6.900   Max.   :2.500  
##  NA's   :25      NA's   :29      NA's   :28      NA's   :31     
##        Species  
##  setosa    :32  
##  versicolor:39  
##  virginica :42  
##  NA's      :37  
##                 
##                 
## 
\end{verbatim}
\end{frame}

\begin{frame}[fragile]{Mengisi Nilai kosong dengan rerata}
\protect\hypertarget{mengisi-nilai-kosong-dengan-rerata}{}
Dari rangkuman diketahui bahwa ada 25 bari yang kosong pada kolom
\texttt{Sepal.Length}

\begin{Shaded}
\begin{Highlighting}[]
\CommentTok{\# mendapatakan rearata kolom Sepal.Length}
\CommentTok{\# mean(iris.mis$Sepal.Length, na.rm = TRUE)}
\NormalTok{iris.mis }\OtherTok{=}\NormalTok{ iris.mis }\SpecialCharTok{\%\textgreater{}\%} 
   \FunctionTok{mutate}\NormalTok{(}\AttributeTok{Sepal.Length =} \FunctionTok{case\_when}\NormalTok{(}
      \FunctionTok{is.na}\NormalTok{(Sepal.Length) }\SpecialCharTok{\textasciitilde{}} \FunctionTok{mean}\NormalTok{(iris.mis}\SpecialCharTok{$}\NormalTok{Sepal.Length, }\AttributeTok{na.rm =} \ConstantTok{TRUE}\NormalTok{), }
      \ConstantTok{TRUE} \SpecialCharTok{\textasciitilde{}}\NormalTok{ Sepal.Length}
\NormalTok{   ))}
\FunctionTok{summary}\NormalTok{(iris.mis)}
\end{Highlighting}
\end{Shaded}

\begin{verbatim}
##   Sepal.Length    Sepal.Width     Petal.Length    Petal.Width   
##  Min.   :4.300   Min.   :2.000   Min.   :1.000   Min.   :0.100  
##  1st Qu.:5.200   1st Qu.:2.800   1st Qu.:1.525   1st Qu.:0.300  
##  Median :5.774   Median :3.000   Median :4.400   Median :1.300  
##  Mean   :5.774   Mean   :3.072   Mean   :3.756   Mean   :1.186  
##  3rd Qu.:6.300   3rd Qu.:3.400   3rd Qu.:5.100   3rd Qu.:1.800  
##  Max.   :7.900   Max.   :4.400   Max.   :6.900   Max.   :2.500  
##                  NA's   :29      NA's   :28      NA's   :31     
##        Species  
##  setosa    :32  
##  versicolor:39  
##  virginica :42  
##  NA's      :37  
##                 
##                 
## 
\end{verbatim}
\end{frame}

\hypertarget{menggunakan-package}{%
\subsection{\texorpdfstring{Menggunakan
\texttt{package}}{Menggunakan package}}\label{menggunakan-package}}

\begin{frame}[fragile]{Menggunakan \texttt{package}}
Untuk melakukan \emph{imputation} kita bisa menggunakan fungsing
\texttt{missForest()} seperti contoh skrip berikut.

\begin{Shaded}
\begin{Highlighting}[]
\NormalTok{iris.imp }\OtherTok{\textless{}{-}} \FunctionTok{missForest}\NormalTok{(iris.mis, }\AttributeTok{xtrue =}\NormalTok{ iris, }\AttributeTok{verbose =} \ConstantTok{TRUE}\NormalTok{)}

\NormalTok{iris.imp}\SpecialCharTok{$}\NormalTok{OOBerror}
\NormalTok{iris.imp}\SpecialCharTok{$}\NormalTok{error}
\NormalTok{result0 }\OtherTok{\textless{}{-}}\NormalTok{ iris.imp}\SpecialCharTok{$}\NormalTok{ximp}

\FunctionTok{summary}\NormalTok{(result0)}
\end{Highlighting}
\end{Shaded}
\end{frame}

\hypertarget{paket-lain-untuk-imputation}{%
\subsection{\texorpdfstring{Paket lain untuk
\emph{imputation}}{Paket lain untuk imputation}}\label{paket-lain-untuk-imputation}}

\begin{frame}[fragile]{Paket lain untuk \emph{imputation}}
Selain dengan menggunakan paket \texttt{missForest}, melakukan
\emph{imputation} juga dapat dilakukan dengan beberapa paket lain dengan
metode dan atau pendekatan yang berbeda. Seperti paket \texttt{Hmisc}
yang juga cukup populer dikalangan pengguna R menawarkan metode
\emph{imputation} dengan beberapa algoritme.

\vspace{0.1in}

Misalnya, dalam paket \texttt{Hmisc} terdapat fungsi \texttt{impute()}
di mana pengguna dapat melakukan \emph{imputation} dengan metode yang
didefinisikan sendiri (contoh: \emph{mean, max, median,} dll). Paket
\texttt{mi} juga bisa digunakan untuk melakukan \emph{imputation} dengan
menggunakan metode \emph{Predictive Mean Matching Method}.
\end{frame}


\section[]{}
\frame{\small \frametitle{Table of Contents}
\tableofcontents}
\end{document}
