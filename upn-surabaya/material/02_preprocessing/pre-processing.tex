\documentclass[10pt,ignorenonframetext,,aspectratio=149]{beamer}
\usefonttheme{serif} % use mainfont rather than sansfont for slide text
\setbeamertemplate{caption}[numbered]
\setbeamertemplate{caption label separator}{: }
\setbeamercolor{caption name}{fg=normal text.fg}
\usepackage{lmodern}
\usepackage{amssymb,amsmath}
\usepackage{ifxetex,ifluatex}
\usepackage{fixltx2e} % provides \textsubscript
\ifnum 0\ifxetex 1\fi\ifluatex 1\fi=0 % if pdftex
  \usepackage[T1]{fontenc}
  \usepackage[utf8]{inputenc}
\else % if luatex or xelatex
  \ifxetex
    \usepackage{mathspec}
  \else
    \usepackage{fontspec}
  \fi
  \defaultfontfeatures{Ligatures=TeX,Scale=MatchLowercase}
  \newcommand{\euro}{€}
    \setmainfont[]{Open Sans}
\fi
% use upquote if available, for straight quotes in verbatim environments
\IfFileExists{upquote.sty}{\usepackage{upquote}}{}
% use microtype if available
\IfFileExists{microtype.sty}{%
\usepackage{microtype}
\UseMicrotypeSet[protrusion]{basicmath} % disable protrusion for tt fonts
}{}
\usepackage{color}
\usepackage{fancyvrb}
\newcommand{\VerbBar}{|}
\newcommand{\VERB}{\Verb[commandchars=\\\{\}]}
\DefineVerbatimEnvironment{Highlighting}{Verbatim}{commandchars=\\\{\}}
% Add ',fontsize=\small' for more characters per line
\usepackage{framed}
\definecolor{shadecolor}{RGB}{248,248,248}
\newenvironment{Shaded}{\begin{snugshade}}{\end{snugshade}}
\newcommand{\AlertTok}[1]{\textcolor[rgb]{0.94,0.16,0.16}{#1}}
\newcommand{\AnnotationTok}[1]{\textcolor[rgb]{0.56,0.35,0.01}{\textbf{\textit{#1}}}}
\newcommand{\AttributeTok}[1]{\textcolor[rgb]{0.77,0.63,0.00}{#1}}
\newcommand{\BaseNTok}[1]{\textcolor[rgb]{0.00,0.00,0.81}{#1}}
\newcommand{\BuiltInTok}[1]{#1}
\newcommand{\CharTok}[1]{\textcolor[rgb]{0.31,0.60,0.02}{#1}}
\newcommand{\CommentTok}[1]{\textcolor[rgb]{0.56,0.35,0.01}{\textit{#1}}}
\newcommand{\CommentVarTok}[1]{\textcolor[rgb]{0.56,0.35,0.01}{\textbf{\textit{#1}}}}
\newcommand{\ConstantTok}[1]{\textcolor[rgb]{0.00,0.00,0.00}{#1}}
\newcommand{\ControlFlowTok}[1]{\textcolor[rgb]{0.13,0.29,0.53}{\textbf{#1}}}
\newcommand{\DataTypeTok}[1]{\textcolor[rgb]{0.13,0.29,0.53}{#1}}
\newcommand{\DecValTok}[1]{\textcolor[rgb]{0.00,0.00,0.81}{#1}}
\newcommand{\DocumentationTok}[1]{\textcolor[rgb]{0.56,0.35,0.01}{\textbf{\textit{#1}}}}
\newcommand{\ErrorTok}[1]{\textcolor[rgb]{0.64,0.00,0.00}{\textbf{#1}}}
\newcommand{\ExtensionTok}[1]{#1}
\newcommand{\FloatTok}[1]{\textcolor[rgb]{0.00,0.00,0.81}{#1}}
\newcommand{\FunctionTok}[1]{\textcolor[rgb]{0.00,0.00,0.00}{#1}}
\newcommand{\ImportTok}[1]{#1}
\newcommand{\InformationTok}[1]{\textcolor[rgb]{0.56,0.35,0.01}{\textbf{\textit{#1}}}}
\newcommand{\KeywordTok}[1]{\textcolor[rgb]{0.13,0.29,0.53}{\textbf{#1}}}
\newcommand{\NormalTok}[1]{#1}
\newcommand{\OperatorTok}[1]{\textcolor[rgb]{0.81,0.36,0.00}{\textbf{#1}}}
\newcommand{\OtherTok}[1]{\textcolor[rgb]{0.56,0.35,0.01}{#1}}
\newcommand{\PreprocessorTok}[1]{\textcolor[rgb]{0.56,0.35,0.01}{\textit{#1}}}
\newcommand{\RegionMarkerTok}[1]{#1}
\newcommand{\SpecialCharTok}[1]{\textcolor[rgb]{0.00,0.00,0.00}{#1}}
\newcommand{\SpecialStringTok}[1]{\textcolor[rgb]{0.31,0.60,0.02}{#1}}
\newcommand{\StringTok}[1]{\textcolor[rgb]{0.31,0.60,0.02}{#1}}
\newcommand{\VariableTok}[1]{\textcolor[rgb]{0.00,0.00,0.00}{#1}}
\newcommand{\VerbatimStringTok}[1]{\textcolor[rgb]{0.31,0.60,0.02}{#1}}
\newcommand{\WarningTok}[1]{\textcolor[rgb]{0.56,0.35,0.01}{\textbf{\textit{#1}}}}

% Comment these out if you don't want a slide with just the
% part/section/subsection/subsubsection title:
\AtBeginPart{
  \let\insertpartnumber\relax
  \let\partname\relax
  \frame{\partpage}
}
\AtBeginSection{
  \let\insertsectionnumber\relax
  \let\sectionname\relax
  \frame{\sectionpage}
}
\AtBeginSubsection{
  \let\insertsubsectionnumber\relax
  \let\subsectionname\relax
  \frame{\subsectionpage}
}

\setlength{\emergencystretch}{3em}  % prevent overfull lines
\providecommand{\tightlist}{%
  \setlength{\itemsep}{0pt}\setlength{\parskip}{0pt}}
\setcounter{secnumdepth}{0}

\title{Pre-processing data di R}
\subtitle{(Pelatihan data sains menggunakan R dan Gephi)}
\author{Ujang Fahmi}
\date{}

%% Here's everything I added.
%%--------------------------

\usepackage{graphicx}
\usepackage{rotating}
%\setbeamertemplate{caption}[numbered]
\usepackage{hyperref}
\usepackage{caption}
\usepackage[normalem]{ulem}
%\mode<presentation>
\usepackage{wasysym}
%\usepackage{amsmath}


% Get rid of navigation symbols.
%-------------------------------
\setbeamertemplate{navigation symbols}{}

% Optional institute tags and titlegraphic.
% Do feel free to change the titlegraphic if you don't want it as a Markdown field.
%----------------------------------------------------------------------------------
\institute{Pelajaran ke-1}

% \titlegraphic{\includegraphics[width=0.3\paperwidth]{\string~/Dropbox/teaching/clemson-academic.png}} % <-- if you want to know what this looks like without it as a Markdown field. 
% -----------------------------------------------------------------------------------------------------
\titlegraphic{\includegraphics[width=0.3\paperwidth]{styles/sadasa.png}}

% Some additional title page adjustments.
%----------------------------------------
\setbeamertemplate{title page}[empty]
%\date{}
\setbeamerfont{subtitle}{size=\small}

\setbeamercovered{transparent}

% Some optional colors. Change or add as you see fit.
%---------------------------------------------------
\definecolor{clemsonpurple}{HTML}{522D80}
 \definecolor{clemsonorange}{HTML}{F66733}
\definecolor{uiucblue}{HTML}{003C7D}
\definecolor{uiucorange}{HTML}{F47F24}


% Some optional color adjustments to Beamer. Change as you see fit.
%------------------------------------------------------------------
\setbeamercolor{frametitle}{fg=clemsonpurple,bg=white}
\setbeamercolor{title}{fg=clemsonpurple,bg=white}
\setbeamercolor{local structure}{fg=clemsonpurple}
\setbeamercolor{section in toc}{fg=clemsonpurple,bg=white}
% \setbeamercolor{subsection in toc}{fg=clemsonorange,bg=white}
\setbeamercolor{footline}{fg=clemsonpurple!50, bg=white}
\setbeamercolor{block title}{fg=clemsonorange,bg=white}


\let\Tiny=\tiny


% Sections and subsections should not get their own damn slide.
%--------------------------------------------------------------
\AtBeginPart{}
\AtBeginSection{}
\AtBeginSubsection{}
\AtBeginSubsubsection{}

% Suppress some of Markdown's weird default vertical spacing.
%------------------------------------------------------------
\setlength{\emergencystretch}{0em}  % prevent overfull lines
\setlength{\parskip}{0pt}


% Allow for those simple two-tone footlines I like. 
% Edit the colors as you see fit.
%--------------------------------------------------
\defbeamertemplate*{footline}{my footline}{%
    \ifnum\insertpagenumber=1
    \hbox{%
        \begin{beamercolorbox}[wd=\paperwidth,ht=.8ex,dp=1ex,center]{}%
      % empty environment to raise height
        \end{beamercolorbox}%
    }%
    \vskip0pt%
    \else%
        \Tiny{%
            \hfill%
		\vspace*{1pt}%
            \insertframenumber/\inserttotalframenumber \hspace*{0.1cm}%
            \newline%
            \color{clemsonpurple}{\rule{\paperwidth}{0.4mm}}\newline%
            \color{clemsonorange}{\rule{\paperwidth}{.4mm}}%
        }%
    \fi%
}

% Various cosmetic things, though I must confess I forget what exactly these do and why I included them.
%-------------------------------------------------------------------------------------------------------
\setbeamercolor{structure}{fg=blue}
\setbeamercolor{local structure}{parent=structure}
\setbeamercolor{item projected}{parent=item,use=item,fg=clemsonpurple,bg=white}
\setbeamercolor{enumerate item}{parent=item}

% Adjust some item elements. More cosmetic things.
%-------------------------------------------------
\setbeamertemplate{itemize item}{\color{clemsonpurple}$\bullet$}
\setbeamertemplate{itemize subitem}{\color{clemsonpurple}\scriptsize{$\bullet$}}
\setbeamertemplate{itemize/enumerate body end}{\vspace{.6\baselineskip}} % So I'm less inclined to use \medskip and \bigskip in Markdown.

% Automatically center images
% ---------------------------
% Note: this is for ![](image.png) images
% Use "fig.align = "center" for R chunks

\usepackage{etoolbox}

\AtBeginDocument{%
  \letcs\oig{@orig\string\includegraphics}%
  \renewcommand<>\includegraphics[2][]{%
    \only#3{%
      {\centering\oig[{#1}]{#2}\par}%
    }%
  }%
}

% I think I've moved to xelatex now. Here's some stuff for that.
% --------------------------------------------------------------
% I could customize/generalize this more but the truth is it works for my circumstances.

\ifxetex
\setbeamerfont{title}{family=\fontspec{Titillium Web}}
\setbeamerfont{frametitle}{family=\fontspec{Titillium Web}}
\usepackage[font=small,skip=0pt]{caption}
 \else
 \fi

% Okay, and begin the actual document...

\begin{document}
\frame{\titlepage}

\begin{frame}
Salam kenal dan selamat datang.

Semoga kita semua bisa saling berbagi pengalaman dan pengetahuan. Saya
adalah Ujang Fahmi, Co-founder dan mentor Sadasa Academy.

Jika anda berada dan sedang membaca tutorial ini, maka kemungkinan anda
adalah orang yang sedang ingin belajar data sains, atau mungkin
ditugaskan untuk mempelajari R oleh institusi atau organisasi anda. Sama
seperti saya dulu, dimana tanpa latar belakang enginering saya
didiharuskan untuk belajar R, demi menyelesaikan tugas akhir dan
akhirnya jadilah seperti saya sekarang ini.

Satu hal yang pasti, ini adalah langkah pertama dari banyak langkah yang
harus dilalui, entah melalui lembaga resmi atau belajar secara mandiri.
Jadi selamat belajar!!!

Ujang Fahmi, Yogyakarta, 2021-09-16
\end{frame}

\hypertarget{mendapatkan-bantuan-dan-mengatasi-error}{%
\section{Mendapatkan Bantuan dan Mengatasi
Error}\label{mendapatkan-bantuan-dan-mengatasi-error}}

\hypertarget{error-yang-umum-1}{%
\subsection{Error yang umum 1}\label{error-yang-umum-1}}

\begin{frame}[fragile]{Error yang umum 1}
Di worksapce kita menulis

\begin{Shaded}
\begin{Highlighting}[]
\NormalTok{a }\SpecialCharTok{+} \DecValTok{198}
\end{Highlighting}
\end{Shaded}

Di console menampilkan:

\texttt{Error:\ object\ \textquotesingle{}a\textquotesingle{}\ not\ found}

Artinya objek \texttt{a} tidak bisa ditemukan sehingga R tidak bisa
menyelesaikan perintah yang diberikan
\end{frame}

\hypertarget{error-yang-umum-2}{%
\subsection{Error yang umum 2}\label{error-yang-umum-2}}

\begin{frame}[fragile]{Error yang umum 2}
Di workspace kita mencoba untuk memberikan perintah impor data dengan
menggunakan fungsi \texttt{read\_csv}, kita juga udah yakin argumen yang
dibutuhkan sudah sesuai.

\begin{Shaded}
\begin{Highlighting}[]
\NormalTok{df1 }\OtherTok{=} \FunctionTok{read\_csv}\NormalTok{(}\StringTok{"folder1/file.csv"}\NormalTok{)}
\end{Highlighting}
\end{Shaded}

Tapi, di \texttt{console} kita masih tetap menerima error berikut:

\texttt{Error\ in\ read\_csv("folder1/file.csv")\ :}

\texttt{could\ not\ find\ function\ "read\_csv"}

Error menunjukkan bahwa fungsi \texttt{read\_csv} tidak ditemukan.
Artinya kita belum memanggil library yang dibutuhkan.
\end{frame}

\hypertarget{impor-dan-ekspor-data}{%
\section{Impor dan Ekspor Data}\label{impor-dan-ekspor-data}}

\begin{frame}[fragile]{Impor dan Ekspor Data}
\begin{quote}
Di R kita hanya bisa mengolah data yang yang sudah kita impor atau
berada di \emph{environment} R. Maka ketika ada
\texttt{error:\ object\ not\ found} di console, sebagian besar
disebabkan oleh data/objek yang belum ada.
\end{quote}
\end{frame}

\hypertarget{impor-data}{%
\subsection{Impor Data}\label{impor-data}}

\begin{frame}[fragile]{Impor Data}
\begin{itemize}
\tightlist
\item
  Impor data merupakan salah satu langkah pertama dan utama dalam
  pengolahan data
\item
  impor umumnya menggunakan sintak dengan awalan \texttt{read}. Misal
  \texttt{read.csv()} atau \texttt{read\_csv}, sama-sama digunakan untuk
  impor data \texttt{.csv}
\end{itemize}
\end{frame}

\begin{frame}[fragile]{Impor data frame}
\protect\hypertarget{impor-data-frame}{}
Data frame atau data flate di R bisa diimpor dengan beberapa sintaks
dari beberapa package.

Argumen yang dibutuhkan:

\begin{enumerate}
\tightlist
\item
  namfolder
\item
  namafile atau link data
\end{enumerate}

\begin{Shaded}
\begin{Highlighting}[]
\CommentTok{\# dasar}
\NormalTok{df1 }\OtherTok{=} \FunctionTok{read.csv}\NormalTok{(}\AttributeTok{file =} \StringTok{"namafolder/namafile.csv"}\NormalTok{, }\AttributeTok{header =} \ConstantTok{TRUE}\NormalTok{, }
               \AttributeTok{sep =} \StringTok{","}\NormalTok{)}
\CommentTok{\# readr}
\FunctionTok{library}\NormalTok{(readr)}
\NormalTok{df2 }\OtherTok{=} \FunctionTok{read\_csv}\NormalTok{(}\AttributeTok{file =} \StringTok{"namafolder/namafile.csv"}\NormalTok{, }\AttributeTok{trim\_ws =} \ConstantTok{TRUE}\NormalTok{)}
\CommentTok{\# xlsx}
\FunctionTok{library}\NormalTok{(readxl)}
\NormalTok{df3 }\OtherTok{=} \FunctionTok{read\_excel}\NormalTok{(}\AttributeTok{path =} \StringTok{"namafolder/namafile.xlsx"}\NormalTok{, }\AttributeTok{sheet =} \DecValTok{1}\NormalTok{)}
\end{Highlighting}
\end{Shaded}
\end{frame}

\begin{frame}[fragile]{Impor data json}
\protect\hypertarget{impor-data-json}{}
\texttt{Json} atau \texttt{java\ script\ object\ notation} adalah format
data yang umum digunakan untuk ditampilkan di sebuah laman/website.
\end{frame}

\begin{frame}{Impor data lain}
\protect\hypertarget{impor-data-lain}{}
\end{frame}

\hypertarget{ekspor-data}{%
\subsection{Ekspor Data}\label{ekspor-data}}

\begin{frame}{Ekspor data frame}
\protect\hypertarget{ekspor-data-frame}{}
\end{frame}

\begin{frame}{Ekspor data lain}
\protect\hypertarget{ekspor-data-lain}{}
\end{frame}

\hypertarget{pengecekan-data}{%
\section{Pengecekan Data}\label{pengecekan-data}}

\hypertarget{jumlah-observasi-dan-variabel}{%
\subsection{Jumlah observasi dan
variabel}\label{jumlah-observasi-dan-variabel}}

\hypertarget{rangkuman-data}{%
\subsection{Rangkuman data}\label{rangkuman-data}}

\hypertarget{penataan-data-data-wrangling}{%
\section{\texorpdfstring{Penataan Data \emph{(Data
Wrangling)}}{Penataan Data (Data Wrangling)}}\label{penataan-data-data-wrangling}}

\hypertarget{memilih-kolom-dan-menyusun-nama-kolom}{%
\subsection{Memilih Kolom dan Menyusun nama
kolom}\label{memilih-kolom-dan-menyusun-nama-kolom}}

\hypertarget{memilih-baris}{%
\subsection{Memilih Baris}\label{memilih-baris}}

\hypertarget{transformasi-kolom}{%
\subsection{Transformasi kolom}\label{transformasi-kolom}}

\begin{frame}{Nama bari menjadi kolom (Row names to column)}
\protect\hypertarget{nama-bari-menjadi-kolom-row-names-to-column}{}
\end{frame}

\begin{frame}{Membuat kolom baru}
\protect\hypertarget{membuat-kolom-baru}{}
\end{frame}

\begin{frame}{Memisah dan Menggabungkan nilai kolom}
\protect\hypertarget{memisah-dan-menggabungkan-nilai-kolom}{}
\end{frame}

\begin{frame}{Merangkum variabel}
\protect\hypertarget{merangkum-variabel}{}
\end{frame}

\hypertarget{mengatasasi-nilai-hilang}{%
\section{Mengatasasi Nilai Hilang}\label{mengatasasi-nilai-hilang}}

\hypertarget{menggunakan-teknik-manual}{%
\subsection{Menggunakan teknik manual}\label{menggunakan-teknik-manual}}

\hypertarget{menggunakan-package}{%
\subsection{\texorpdfstring{Menggunakan
\texttt{package}}{Menggunakan package}}\label{menggunakan-package}}


\section[]{}
\frame{\small \frametitle{Table of Contents}
\tableofcontents}
\end{document}
