\documentclass[10pt,ignorenonframetext,,aspectratio=149]{beamer}
\usefonttheme{serif} % use mainfont rather than sansfont for slide text
\setbeamertemplate{caption}[numbered]
\setbeamertemplate{caption label separator}{: }
\setbeamercolor{caption name}{fg=normal text.fg}
\usepackage{lmodern}
\usepackage{amssymb,amsmath}
\usepackage{ifxetex,ifluatex}
\usepackage{fixltx2e} % provides \textsubscript
\ifnum 0\ifxetex 1\fi\ifluatex 1\fi=0 % if pdftex
  \usepackage[T1]{fontenc}
  \usepackage[utf8]{inputenc}
\else % if luatex or xelatex
  \ifxetex
    \usepackage{mathspec}
  \else
    \usepackage{fontspec}
  \fi
  \defaultfontfeatures{Ligatures=TeX,Scale=MatchLowercase}
  \newcommand{\euro}{€}
    \setmainfont[]{Open Sans}
\fi
% use upquote if available, for straight quotes in verbatim environments
\IfFileExists{upquote.sty}{\usepackage{upquote}}{}
% use microtype if available
\IfFileExists{microtype.sty}{%
\usepackage{microtype}
\UseMicrotypeSet[protrusion]{basicmath} % disable protrusion for tt fonts
}{}
\usepackage{color}
\usepackage{fancyvrb}
\newcommand{\VerbBar}{|}
\newcommand{\VERB}{\Verb[commandchars=\\\{\}]}
\DefineVerbatimEnvironment{Highlighting}{Verbatim}{commandchars=\\\{\}}
% Add ',fontsize=\small' for more characters per line
\usepackage{framed}
\definecolor{shadecolor}{RGB}{248,248,248}
\newenvironment{Shaded}{\begin{snugshade}}{\end{snugshade}}
\newcommand{\AlertTok}[1]{\textcolor[rgb]{0.94,0.16,0.16}{#1}}
\newcommand{\AnnotationTok}[1]{\textcolor[rgb]{0.56,0.35,0.01}{\textbf{\textit{#1}}}}
\newcommand{\AttributeTok}[1]{\textcolor[rgb]{0.77,0.63,0.00}{#1}}
\newcommand{\BaseNTok}[1]{\textcolor[rgb]{0.00,0.00,0.81}{#1}}
\newcommand{\BuiltInTok}[1]{#1}
\newcommand{\CharTok}[1]{\textcolor[rgb]{0.31,0.60,0.02}{#1}}
\newcommand{\CommentTok}[1]{\textcolor[rgb]{0.56,0.35,0.01}{\textit{#1}}}
\newcommand{\CommentVarTok}[1]{\textcolor[rgb]{0.56,0.35,0.01}{\textbf{\textit{#1}}}}
\newcommand{\ConstantTok}[1]{\textcolor[rgb]{0.00,0.00,0.00}{#1}}
\newcommand{\ControlFlowTok}[1]{\textcolor[rgb]{0.13,0.29,0.53}{\textbf{#1}}}
\newcommand{\DataTypeTok}[1]{\textcolor[rgb]{0.13,0.29,0.53}{#1}}
\newcommand{\DecValTok}[1]{\textcolor[rgb]{0.00,0.00,0.81}{#1}}
\newcommand{\DocumentationTok}[1]{\textcolor[rgb]{0.56,0.35,0.01}{\textbf{\textit{#1}}}}
\newcommand{\ErrorTok}[1]{\textcolor[rgb]{0.64,0.00,0.00}{\textbf{#1}}}
\newcommand{\ExtensionTok}[1]{#1}
\newcommand{\FloatTok}[1]{\textcolor[rgb]{0.00,0.00,0.81}{#1}}
\newcommand{\FunctionTok}[1]{\textcolor[rgb]{0.00,0.00,0.00}{#1}}
\newcommand{\ImportTok}[1]{#1}
\newcommand{\InformationTok}[1]{\textcolor[rgb]{0.56,0.35,0.01}{\textbf{\textit{#1}}}}
\newcommand{\KeywordTok}[1]{\textcolor[rgb]{0.13,0.29,0.53}{\textbf{#1}}}
\newcommand{\NormalTok}[1]{#1}
\newcommand{\OperatorTok}[1]{\textcolor[rgb]{0.81,0.36,0.00}{\textbf{#1}}}
\newcommand{\OtherTok}[1]{\textcolor[rgb]{0.56,0.35,0.01}{#1}}
\newcommand{\PreprocessorTok}[1]{\textcolor[rgb]{0.56,0.35,0.01}{\textit{#1}}}
\newcommand{\RegionMarkerTok}[1]{#1}
\newcommand{\SpecialCharTok}[1]{\textcolor[rgb]{0.00,0.00,0.00}{#1}}
\newcommand{\SpecialStringTok}[1]{\textcolor[rgb]{0.31,0.60,0.02}{#1}}
\newcommand{\StringTok}[1]{\textcolor[rgb]{0.31,0.60,0.02}{#1}}
\newcommand{\VariableTok}[1]{\textcolor[rgb]{0.00,0.00,0.00}{#1}}
\newcommand{\VerbatimStringTok}[1]{\textcolor[rgb]{0.31,0.60,0.02}{#1}}
\newcommand{\WarningTok}[1]{\textcolor[rgb]{0.56,0.35,0.01}{\textbf{\textit{#1}}}}
\usepackage{graphicx,grffile}
\makeatletter
\def\maxwidth{\ifdim\Gin@nat@width>\linewidth\linewidth\else\Gin@nat@width\fi}
\def\maxheight{\ifdim\Gin@nat@height>\textheight0.8\textheight\else\Gin@nat@height\fi}
\makeatother
% Scale images if necessary, so that they will not overflow the page
% margins by default, and it is still possible to overwrite the defaults
% using explicit options in \includegraphics[width, height, ...]{}
\setkeys{Gin}{width=\maxwidth,height=\maxheight,keepaspectratio}

% Comment these out if you don't want a slide with just the
% part/section/subsection/subsubsection title:
\AtBeginPart{
  \let\insertpartnumber\relax
  \let\partname\relax
  \frame{\partpage}
}
\AtBeginSection{
  \let\insertsectionnumber\relax
  \let\sectionname\relax
  \frame{\sectionpage}
}
\AtBeginSubsection{
  \let\insertsubsectionnumber\relax
  \let\subsectionname\relax
  \frame{\subsectionpage}
}

\setlength{\emergencystretch}{3em}  % prevent overfull lines
\providecommand{\tightlist}{%
  \setlength{\itemsep}{0pt}\setlength{\parskip}{0pt}}
\setcounter{secnumdepth}{0}

\title{Visualisasi data menggunakan R}
\subtitle{(A Subtitle Would Go Here if This Were a Class)}
\author{Ujang Fahmi}
\date{}

%% Here's everything I added.
%%--------------------------

\usepackage{graphicx}
\usepackage{rotating}
%\setbeamertemplate{caption}[numbered]
\usepackage{hyperref}
\usepackage{caption}
\usepackage[normalem]{ulem}
%\mode<presentation>
\usepackage{wasysym}
%\usepackage{amsmath}


% Get rid of navigation symbols.
%-------------------------------
\setbeamertemplate{navigation symbols}{}

% Optional institute tags and titlegraphic.
% Do feel free to change the titlegraphic if you don't want it as a Markdown field.
%----------------------------------------------------------------------------------
\institute{Pelajaran ke-4}

% \titlegraphic{\includegraphics[width=0.3\paperwidth]{\string~/Dropbox/teaching/clemson-academic.png}} % <-- if you want to know what this looks like without it as a Markdown field. 
% -----------------------------------------------------------------------------------------------------
\titlegraphic{\includegraphics[width=0.3\paperwidth]{styles/sadasa.png}}

% Some additional title page adjustments.
%----------------------------------------
\setbeamertemplate{title page}[empty]
%\date{}
\setbeamerfont{subtitle}{size=\small}

\setbeamercovered{transparent}

% Some optional colors. Change or add as you see fit.
%---------------------------------------------------
\definecolor{clemsonpurple}{HTML}{522D80}
 \definecolor{clemsonorange}{HTML}{F66733}
\definecolor{uiucblue}{HTML}{003C7D}
\definecolor{uiucorange}{HTML}{F47F24}


% Some optional color adjustments to Beamer. Change as you see fit.
%------------------------------------------------------------------
\setbeamercolor{frametitle}{fg=clemsonpurple,bg=white}
\setbeamercolor{title}{fg=clemsonpurple,bg=white}
\setbeamercolor{local structure}{fg=clemsonpurple}
\setbeamercolor{section in toc}{fg=clemsonpurple,bg=white}
% \setbeamercolor{subsection in toc}{fg=clemsonorange,bg=white}
\setbeamercolor{footline}{fg=clemsonpurple!50, bg=white}
\setbeamercolor{block title}{fg=clemsonorange,bg=white}


\let\Tiny=\tiny


% Sections and subsections should not get their own damn slide.
%--------------------------------------------------------------
\AtBeginPart{}
\AtBeginSection{}
\AtBeginSubsection{}
\AtBeginSubsubsection{}

% Suppress some of Markdown's weird default vertical spacing.
%------------------------------------------------------------
\setlength{\emergencystretch}{0em}  % prevent overfull lines
\setlength{\parskip}{0pt}


% Allow for those simple two-tone footlines I like. 
% Edit the colors as you see fit.
%--------------------------------------------------
\defbeamertemplate*{footline}{my footline}{%
    \ifnum\insertpagenumber=1
    \hbox{%
        \begin{beamercolorbox}[wd=\paperwidth,ht=.8ex,dp=1ex,center]{}%
      % empty environment to raise height
        \end{beamercolorbox}%
    }%
    \vskip0pt%
    \else%
        \Tiny{%
            \hfill%
		\vspace*{1pt}%
            \insertframenumber/\inserttotalframenumber \hspace*{0.1cm}%
            \newline%
            \color{clemsonpurple}{\rule{\paperwidth}{0.4mm}}\newline%
            \color{clemsonorange}{\rule{\paperwidth}{.4mm}}%
        }%
    \fi%
}

% Various cosmetic things, though I must confess I forget what exactly these do and why I included them.
%-------------------------------------------------------------------------------------------------------
\setbeamercolor{structure}{fg=blue}
\setbeamercolor{local structure}{parent=structure}
\setbeamercolor{item projected}{parent=item,use=item,fg=clemsonpurple,bg=white}
\setbeamercolor{enumerate item}{parent=item}

% Adjust some item elements. More cosmetic things.
%-------------------------------------------------
\setbeamertemplate{itemize item}{\color{clemsonpurple}$\bullet$}
\setbeamertemplate{itemize subitem}{\color{clemsonpurple}\scriptsize{$\bullet$}}
\setbeamertemplate{itemize/enumerate body end}{\vspace{.6\baselineskip}} % So I'm less inclined to use \medskip and \bigskip in Markdown.

% Automatically center images
% ---------------------------
% Note: this is for ![](image.png) images
% Use "fig.align = "center" for R chunks

\usepackage{etoolbox}

\AtBeginDocument{%
  \letcs\oig{@orig\string\includegraphics}%
  \renewcommand<>\includegraphics[2][]{%
    \only#3{%
      {\centering\oig[{#1}]{#2}\par}%
    }%
  }%
}

% I think I've moved to xelatex now. Here's some stuff for that.
% --------------------------------------------------------------
% I could customize/generalize this more but the truth is it works for my circumstances.

\ifxetex
\setbeamerfont{title}{family=\fontspec{Titillium Web}}
\setbeamerfont{frametitle}{family=\fontspec{Titillium Web}}
\usepackage[font=small,skip=0pt]{caption}
 \else
 \fi

% Okay, and begin the actual document...

\begin{document}
\frame{\titlepage}

\begin{frame}
Salam kenal dan selamat datang.

Semoga kita semua bisa saling berbagi pengalaman dan pengetahuan. Saya
adalah Ujang Fahmi, Co-founder dan mentor Sadasa Academy.

\vspace{0.1in}

Jika anda berada dan sedang membaca tutorial ini, maka kemungkinan anda
adalah orang yang sedang ingin belajar data sains, atau mungkin
ditugaskan untuk mempelajari R oleh institusi atau organisasi anda. Sama
seperti saya dulu, dimana tanpa latar belakang enginering saya
didiharuskan untuk belajar R, demi menyelesaikan tugas akhir dan
akhirnya jadilah seperti saya sekarang ini.

\vspace{0.1in}

Satu hal yang pasti, ini adalah langkah pertama dari banyak langkah yang
harus dilalui, entah melalui lembaga resmi atau belajar secara mandiri.
Jadi selamat belajar!!!

\vspace{0.1in}

Ujang Fahmi,

Yogyakarta, 2021-10-09

\vspace{0.1in}

\emph{Materi yang disampaikan disimpan dan dokumentasikan}
\href{https://github.com/eppofahmi/belajaR/tree/master/upn-surabaya}{\textbf{disini}}
\end{frame}

\hypertarget{menggunakan-ggplot}{%
\section{\texorpdfstring{Menggunakan
\texttt{ggplot}}{Menggunakan ggplot}}\label{menggunakan-ggplot}}

\hypertarget{apa-gg-dalam-ggplot}{%
\subsection{\texorpdfstring{Apa \texttt{gg} dalam
\texttt{ggplot}?}{Apa gg dalam ggplot?}}\label{apa-gg-dalam-ggplot}}

\begin{frame}[fragile]{Apa \texttt{gg} dalam \texttt{ggplot}?}
\texttt{gg} merupakan singkatan dari grammar of graphic, yaitu sebuah
aturan pembuatan plot yang setidaknya terdiri dari dua bagian yaitu:

\begin{enumerate}
\item
  Data yang akan diplotkan di sumbu x dan y, disebut aesthetic
  (\texttt{aes()}) di \texttt{ggplot2}
\item
  Geometry atau geom, untuk menentukan jenis plot yang akan dibuat
\end{enumerate}
\end{frame}

\hypertarget{menyiapkan-data}{%
\subsection{Menyiapkan data}\label{menyiapkan-data}}

\begin{frame}[fragile]{Menyiapkan data}
Setelah kita tahu \texttt{gg} selanjutnya untuk bisa membuat plot kita
juga harus tahu data. Data ini yang kemudian akan ditempakan sebagai
sumbu x dan y. Oleh karena itu, hal yang perlu disiapkan ketika akan
membuat plot adalah \texttt{data\ yang\ akan\ diplotkan}.
\end{frame}

\hypertarget{menentukan-sumu-x-y-dan-z}{%
\subsection{Menentukan sumu x, y dan
z}\label{menentukan-sumu-x-y-dan-z}}

\begin{frame}[fragile]{Menentukan sumu x, y dan z}
Di dalam ggplot, kita tidak hanya bisa menempatakn sumbu \texttt{x} dan
\texttt{y.} Tapi kita bisa menempatkan data ketiga, yang biasa disebut
saja sumbu \texttt{z}. Sumbu z ditempatkan sesuai dengan jenis plot yang
akan dibuat.

Sumbu \texttt{z} yang paling sering digunakan di \texttt{ggplot2} ada
dua yaitu:

\begin{enumerate}
\tightlist
\item
  \texttt{color} atau \texttt{colour} untuk plot dengan basis
  \texttt{line}
\item
  \texttt{fill} untuk plot dengan basis kotak atau lingkaran
\end{enumerate}
\end{frame}

\hypertarget{memilih-jenis-plot}{%
\subsection{Memilih jenis plot}\label{memilih-jenis-plot}}

\begin{frame}[fragile]{Memilih jenis plot}
Untuk bisa memilih jenis plot kita bisa menggunakan sintaks
\texttt{geom\_*}, dimana \texttt{*} akan dijadikan sebagai jenis
plotnya. Misalnya:,

\begin{enumerate}
\tightlist
\item
  Untuk membuat plot line kita bisa menggunakan \texttt{geom\_line()}
\item
  Untuk membuat plot column kita bisa menggunakan \texttt{geom\_col()}
\item
  Untuk membuat plot bar kita bisa menggunakan \texttt{geom\_bar()}
\end{enumerate}
\end{frame}

\hypertarget{menyimpan-plot}{%
\subsection{Menyimpan plot}\label{menyimpan-plot}}

\begin{frame}[fragile]{Menyimpan plot}
Untuk menyimpan plot, kita bisa menggunakan tombol export atau
menggunakan sintaks perintah \texttt{ggsave()}.

\begin{Shaded}
\begin{Highlighting}[]
\FunctionTok{ggsave}\NormalTok{(}\StringTok{"mtcars.pdf"}\NormalTok{)}
\FunctionTok{ggsave}\NormalTok{(}\StringTok{"mtcars.png"}\NormalTok{)}

\FunctionTok{ggsave}\NormalTok{(}\StringTok{"mtcars.pdf"}\NormalTok{, }\AttributeTok{width =} \DecValTok{4}\NormalTok{, }\AttributeTok{height =} \DecValTok{4}\NormalTok{)}
\FunctionTok{ggsave}\NormalTok{(}\StringTok{"mtcars.pdf"}\NormalTok{, }\AttributeTok{width =} \DecValTok{20}\NormalTok{, }\AttributeTok{height =} \DecValTok{20}\NormalTok{, }\AttributeTok{units =} \StringTok{"cm"}\NormalTok{)}
\end{Highlighting}
\end{Shaded}
\end{frame}

\hypertarget{jenis-plot}{%
\section{Jenis Plot}\label{jenis-plot}}

\begin{frame}{Jenis Plot}
\begin{figure}
\centering
\includegraphics{images/data-chart-type.png}
\caption{Jenis polot yang umum digunakan}
\end{figure}
\end{frame}

\hypertarget{plot-distribusi}{%
\subsection{Plot Distribusi}\label{plot-distribusi}}

\begin{frame}{Plot Distribusi}
\begin{quote}
Plot distribusi biasanya dibuat menggunakan baris/line untuk menunjukkan
perubahan nilai dalam sebuah rentang data.
\end{quote}
\end{frame}

\begin{frame}[fragile]{Line Plot}
\protect\hypertarget{line-plot}{}
\begin{columns}[T]
\begin{column}{0.5\textwidth}
\begin{Shaded}
\begin{Highlighting}[]
\FunctionTok{library}\NormalTok{(tidyverse)}

\NormalTok{df1 }\OtherTok{=} \FunctionTok{sample\_n}\NormalTok{(economics\_long, }\DecValTok{100}\NormalTok{)}
\NormalTok{df1 }\SpecialCharTok{\%\textgreater{}\%} 
   \FunctionTok{ggplot}\NormalTok{(}\FunctionTok{aes}\NormalTok{(}\AttributeTok{x =}\NormalTok{ date, }
              \AttributeTok{y =}\NormalTok{ value, }
              \AttributeTok{color =}\NormalTok{ variable)) }\SpecialCharTok{+} 
   \FunctionTok{geom\_line}\NormalTok{()}
\end{Highlighting}
\end{Shaded}
\end{column}

\begin{column}{0.5\textwidth}
Contoh plot line:

\includegraphics{figs/unnamed-chunk-3.pdf}
\end{column}
\end{columns}
\end{frame}

\begin{frame}[fragile]{Box-plot}
\protect\hypertarget{box-plot}{}
\begin{columns}[T]
\begin{column}{0.5\textwidth}
\begin{Shaded}
\begin{Highlighting}[]
\NormalTok{mtcars }\SpecialCharTok{\%\textgreater{}\%} 
   \FunctionTok{ggplot}\NormalTok{(}\FunctionTok{aes}\NormalTok{(}\AttributeTok{x=}\FunctionTok{as.factor}\NormalTok{(cyl), }
              \AttributeTok{y=}\NormalTok{mpg)) }\SpecialCharTok{+} 
    \FunctionTok{geom\_boxplot}\NormalTok{(}\AttributeTok{fill=}\StringTok{"slateblue"}\NormalTok{, }
                 \AttributeTok{alpha=}\FloatTok{0.2}\NormalTok{) }\SpecialCharTok{+} 
    \FunctionTok{xlab}\NormalTok{(}\StringTok{"cyl"}\NormalTok{)}
\end{Highlighting}
\end{Shaded}
\end{column}

\begin{column}{0.5\textwidth}
Contoh boxplot:

\includegraphics{figs/unnamed-chunk-5.pdf}
\end{column}
\end{columns}
\end{frame}

\begin{frame}[fragile]{Histogram}
\protect\hypertarget{histogram}{}
\begin{columns}[T]
\begin{column}{0.5\textwidth}
\begin{Shaded}
\begin{Highlighting}[]
\FunctionTok{library}\NormalTok{(hrbrthemes)}

\NormalTok{data }\OtherTok{\textless{}{-}}\NormalTok{ mtcars}

\NormalTok{data }\SpecialCharTok{\%\textgreater{}\%}
  \FunctionTok{ggplot}\NormalTok{(}\FunctionTok{aes}\NormalTok{(}\AttributeTok{x=}\NormalTok{mpg)) }\SpecialCharTok{+}
    \FunctionTok{geom\_histogram}\NormalTok{(}\AttributeTok{binwidth=}\DecValTok{5}\NormalTok{, }
                   \AttributeTok{fill=}\StringTok{"\#69b3a2"}\NormalTok{, }
                   \AttributeTok{color=}\StringTok{"\#e9ecef"}\NormalTok{, }
                   \AttributeTok{alpha=}\FloatTok{0.9}\NormalTok{) }\SpecialCharTok{+}
    \FunctionTok{ggtitle}\NormalTok{(}\StringTok{"Bin size = 5"}\NormalTok{)}
\end{Highlighting}
\end{Shaded}
\end{column}

\begin{column}{0.5\textwidth}
Contoh histogram:

\includegraphics{figs/unnamed-chunk-7.pdf}
\end{column}
\end{columns}
\end{frame}

\hypertarget{plot-perbandingan}{%
\subsection{Plot Perbandingan}\label{plot-perbandingan}}

\begin{frame}[fragile]{Bar}
\protect\hypertarget{bar}{}
\begin{columns}[T]
\begin{column}{0.5\textwidth}
\begin{Shaded}
\begin{Highlighting}[]
\NormalTok{df }\OtherTok{\textless{}{-}} \FunctionTok{data.frame}\NormalTok{(}
   \AttributeTok{dose=}\FunctionTok{c}\NormalTok{(}\StringTok{"D0.5"}\NormalTok{, }\StringTok{"D1"}\NormalTok{, }\StringTok{"D2"}\NormalTok{),}
   \AttributeTok{len=}\FunctionTok{c}\NormalTok{(}\FloatTok{4.2}\NormalTok{, }\DecValTok{10}\NormalTok{, }\FloatTok{29.5}\NormalTok{))}

\NormalTok{df }\SpecialCharTok{\%\textgreater{}\%} 
   \FunctionTok{ggplot}\NormalTok{(}\FunctionTok{aes}\NormalTok{(}\AttributeTok{x=}\NormalTok{dose, }\AttributeTok{y=}\NormalTok{len)) }\SpecialCharTok{+}
   \FunctionTok{geom\_bar}\NormalTok{(}\AttributeTok{stat=}\StringTok{"identity"}\NormalTok{)}
\end{Highlighting}
\end{Shaded}
\end{column}

\begin{column}{0.5\textwidth}
Contoh bar:

\includegraphics{figs/unnamed-chunk-9.pdf}
\end{column}
\end{columns}
\end{frame}

\begin{frame}[fragile]{Column}
\protect\hypertarget{column}{}
\begin{columns}[T]
\begin{column}{0.5\textwidth}
\begin{Shaded}
\begin{Highlighting}[]
\NormalTok{df2 }\OtherTok{\textless{}{-}} \FunctionTok{data.frame}\NormalTok{(}
   \AttributeTok{supp=}\FunctionTok{rep}\NormalTok{(}\FunctionTok{c}\NormalTok{(}\StringTok{"VC"}\NormalTok{, }\StringTok{"OJ"}\NormalTok{), }\AttributeTok{each=}\DecValTok{3}\NormalTok{),}
   \AttributeTok{dose=}\FunctionTok{rep}\NormalTok{(}\FunctionTok{c}\NormalTok{(}\StringTok{"D0.5"}\NormalTok{, }\StringTok{"D1"}\NormalTok{, }\StringTok{"D2"}\NormalTok{),}\DecValTok{2}\NormalTok{),}
   \AttributeTok{len=}\FunctionTok{c}\NormalTok{(}\FloatTok{6.8}\NormalTok{, }\DecValTok{15}\NormalTok{, }\DecValTok{33}\NormalTok{, }\FloatTok{4.2}\NormalTok{, }\DecValTok{10}\NormalTok{, }\FloatTok{29.5}\NormalTok{))}

\NormalTok{df2 }\SpecialCharTok{\%\textgreater{}\%} 
   \FunctionTok{ggplot}\NormalTok{(}\FunctionTok{aes}\NormalTok{(}\AttributeTok{x=}\NormalTok{dose, }
              \AttributeTok{y=}\NormalTok{len, }
              \AttributeTok{fill=}\NormalTok{supp)) }\SpecialCharTok{+}
   \FunctionTok{geom\_col}\NormalTok{(}\AttributeTok{position=}\FunctionTok{position\_dodge}\NormalTok{())}\SpecialCharTok{+}
   \FunctionTok{geom\_text}\NormalTok{(}\FunctionTok{aes}\NormalTok{(}\AttributeTok{label=}\NormalTok{len), }
            \AttributeTok{vjust=}\FloatTok{1.6}\NormalTok{, }
            \AttributeTok{color=}\StringTok{"white"}\NormalTok{,}
            \AttributeTok{position =} \FunctionTok{position\_dodge}\NormalTok{(}\FloatTok{0.9}\NormalTok{),}
            \AttributeTok{size=}\FloatTok{3.5}\NormalTok{)}\SpecialCharTok{+}
   \FunctionTok{scale\_fill\_brewer}\NormalTok{(}\AttributeTok{palette=}\StringTok{"Paired"}\NormalTok{)}\SpecialCharTok{+}
   \FunctionTok{theme\_minimal}\NormalTok{()}
\end{Highlighting}
\end{Shaded}
\end{column}

\begin{column}{0.5\textwidth}
Contoh column:

\includegraphics{figs/unnamed-chunk-11.pdf}
\end{column}
\end{columns}
\end{frame}

\hypertarget{plot-hubungan}{%
\subsection{Plot Hubungan}\label{plot-hubungan}}

\begin{frame}[fragile]{Plot Hubungan}
\begin{columns}[T]
\begin{column}{0.5\textwidth}
\begin{Shaded}
\begin{Highlighting}[]
\NormalTok{mtcars }\SpecialCharTok{\%\textgreater{}\%} 
   \FunctionTok{ggplot}\NormalTok{(}\FunctionTok{aes}\NormalTok{(}\AttributeTok{x =}\NormalTok{ mpg, }
              \AttributeTok{y =}\NormalTok{ drat)) }\SpecialCharTok{+}
   \FunctionTok{geom\_point}\NormalTok{(}
      \FunctionTok{aes}\NormalTok{(}
         \AttributeTok{color =} 
            \FunctionTok{factor}\NormalTok{(gear)}
\NormalTok{         )}
\NormalTok{      )}
\end{Highlighting}
\end{Shaded}
\end{column}

\begin{column}{0.5\textwidth}
Contoh plot hubungan:

\includegraphics{figs/unnamed-chunk-13.pdf}
\end{column}
\end{columns}
\end{frame}

\hypertarget{plot-komposisi}{%
\subsection{Plot Komposisi}\label{plot-komposisi}}

\begin{frame}[fragile]{Pie chart dasar}
\protect\hypertarget{pie-chart-dasar}{}
\begin{columns}[T]
\begin{column}{0.5\textwidth}
\begin{Shaded}
\begin{Highlighting}[]
\NormalTok{data }\OtherTok{\textless{}{-}} \FunctionTok{data.frame}\NormalTok{(}
  \AttributeTok{group=}\NormalTok{LETTERS[}\DecValTok{1}\SpecialCharTok{:}\DecValTok{5}\NormalTok{],}
  \AttributeTok{value=}\FunctionTok{c}\NormalTok{(}\DecValTok{13}\NormalTok{,}\DecValTok{7}\NormalTok{,}\DecValTok{9}\NormalTok{,}\DecValTok{21}\NormalTok{,}\DecValTok{2}\NormalTok{)}
\NormalTok{)}

\CommentTok{\# Basic piechart}
\NormalTok{data }\SpecialCharTok{\%\textgreater{}\%} 
   \FunctionTok{ggplot}\NormalTok{(}\FunctionTok{aes}\NormalTok{(}\AttributeTok{x=}\StringTok{""}\NormalTok{,}
              \AttributeTok{y=}\NormalTok{value,}
              \AttributeTok{fill=}\NormalTok{group)) }\SpecialCharTok{+}
   \FunctionTok{geom\_bar}\NormalTok{(}\AttributeTok{stat=}\StringTok{"identity"}\NormalTok{, }
            \AttributeTok{width=}\DecValTok{1}\NormalTok{) }\SpecialCharTok{+}
   \FunctionTok{coord\_polar}\NormalTok{(}\StringTok{"y"}\NormalTok{, }\AttributeTok{start=}\DecValTok{0}\NormalTok{)}
\end{Highlighting}
\end{Shaded}
\end{column}

\begin{column}{0.5\textwidth}
Contoh pie plot 1:

\includegraphics{figs/unnamed-chunk-15.pdf}
\end{column}
\end{columns}
\end{frame}

\begin{frame}[fragile]{Memperbaiki pie chart 1}
\protect\hypertarget{memperbaiki-pie-chart-1}{}
\begin{columns}[T]
\begin{column}{0.5\textwidth}
\begin{Shaded}
\begin{Highlighting}[]
\NormalTok{data }\SpecialCharTok{\%\textgreater{}\%} 
   \FunctionTok{ggplot}\NormalTok{(}\FunctionTok{aes}\NormalTok{(}\AttributeTok{x=}\StringTok{""}\NormalTok{, }
              \AttributeTok{y=}\NormalTok{value, }
              \AttributeTok{fill=}\NormalTok{group)) }\SpecialCharTok{+}
   \FunctionTok{geom\_bar}\NormalTok{(}\AttributeTok{stat=}\StringTok{"identity"}\NormalTok{, }
            \AttributeTok{width=}\DecValTok{1}\NormalTok{, }
            \AttributeTok{color=}\StringTok{"white"}\NormalTok{) }\SpecialCharTok{+}
   \FunctionTok{coord\_polar}\NormalTok{(}\StringTok{"y"}\NormalTok{, }\AttributeTok{start=}\DecValTok{0}\NormalTok{) }\SpecialCharTok{+}
   \FunctionTok{theme\_void}\NormalTok{()}
\end{Highlighting}
\end{Shaded}
\end{column}

\begin{column}{0.5\textwidth}
Contoh pie chart 2:

\includegraphics{figs/unnamed-chunk-17.pdf}
\end{column}
\end{columns}
\end{frame}

\begin{frame}{Memperbaiki pie chart 2}
\protect\hypertarget{memperbaiki-pie-chart-2}{}
\includegraphics{figs/unnamed-chunk-18.pdf}
\end{frame}

\hypertarget{your-turn}{%
\section{Your Turn}\label{your-turn}}

\begin{frame}{Your Turn}
\begin{enumerate}
\tightlist
\item
  Membuat plot perbandingan
\item
  Membuat plot komposisi
\item
  Membuat plot hubungan
\item
  Membuat plot distribusi
\end{enumerate}
\end{frame}


\section[]{}
\frame{\small \frametitle{Table of Contents}
\tableofcontents}
\end{document}
